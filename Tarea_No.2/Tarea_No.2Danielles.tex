\documentclass[10pt,a4paper,openany]{article}
\usepackage[latin1]{inputenc}
\usepackage{amsmath}
\usepackage{amsfonts}
\usepackage{amssymb}
\usepackage{graphicx}
\usepackage{listings}
\usepackage{color}
\usepackage[left=2cm,right=2cm,top=3cm,bottom=2cm]{geometry}
\usepackage[numbers,sort&compress]{natbib}
\usepackage[spanish]{babel}
\usepackage{caption}
\usepackage{url}


\setlength{\parindent}{12pt}

\definecolor{mygreen}{rgb}{0,0.6,0}
\definecolor{mygray}{rgb}{0.5,0.5,0.5}
\definecolor{mymauve}{rgb}{0.58,0,0.82}

\lstset{ 
	backgroundcolor=\color{white},   % choose the background color; you must add \usepackage{color} or \usepackage{xcolor}; should come as last argument
	basicstyle=\footnotesize,        % the size of the fonts that are used for the code
	breakatwhitespace=false,         % sets if automatic breaks should only happen at whitespace
	breaklines=true,                 % sets automatic line breaking
	captionpos=b,                    % sets the caption-position to bottom
	commentstyle=\color{mygreen},    % comment style
	deletekeywords={...},            % if you want to delete keywords from the given language
	escapeinside={\%*}{*)},          % if you want to add LaTeX within your code
	extendedchars=true,              % lets you use non-ASCII characters; for 8-bits encodings only, does not work with UTF-8
	firstnumber=01,                	 % start line enumeration with line 1000
	frame=single,	                 % adds a frame around the code
	keepspaces=true,                 % keeps spaces in text, useful for keeping indentation of code (possibly needs columns=flexible)
	keywordstyle=\color{blue},       % keyword style
	language=Python,                 % the language of the code
	morekeywords={*,...},            % if you want to add more keywords to the set
	numbers=left,                    % where to put the line-numbers; possible values are (none, left, right)
	numbersep=5pt,                   % how far the line-numbers are from the code
	numberstyle=\tiny\color{mygray}, % the style that is used for the line-numbers
	rulecolor=\color{black},         % if not set, the frame-color may be changed on line-breaks within not-black text (e.g. comments (green here))
	showspaces=false,                % show spaces everywhere adding particular underscores; it overrides 'showstringspaces'
	showstringspaces=false,          % underline spaces within strings only
	showtabs=false,                  % show tabs within strings adding particular underscores
	stepnumber=1,                    % the step between two line-numbers. If it's 1, each line will be numbered
	stringstyle=\color{mymauve},     % string literal style
	tabsize=2,	                     % sets default tabsize to 2 spaces
	title=\lstname                   % show the filename of files included with \lstinputlisting; also try caption instead of title
}
\title{Tarea No.2}
\author{5272}
\date{}


\begin{document}
	\maketitle
	
	
	
	\section{Circular Layout}
	
	Este algoritmo de dise�o como su nombre lo indica puede ser �til cuando se desea estructurar la informaci�n con un patr�n circular y se trata de disminuir los cruces entre los v�rtices \citep{belviranli2009circular}; es aplicado en redes sociales y administraci�n de redes, con este se puede realizar estructuras de grupos y �rbol dentro de una red \citep{gansner2006improved}.
	\lstinputlisting[language=Python]{Tarea2_01.py}
		\begin{center}
		\includegraphics[scale=0.7]{Tarea2_01.eps}	
		\captionof{figure}{Circular Layout}
	    \end{center}
   
	\newpage
	\section{Kamada Kawai Layout}	
	
	Este algoritmo es para grafos conectados y no dirigidos; se relaciona con un sistema de resorte din�mico donde la fuerza entre dos v�rtices es inversamente proporcional al cuadrado de la distancia entre estos ,es decir, los v�rtices que tienen los resortes m�s fuertes se encuentran m�s cerca. \citep{kamada1989algorithm}.
	\lstinputlisting[language=Python]{Tarea2_02.py}
	\begin{center}
		\includegraphics[scale=0.7]{Tarea2_02.eps}	
		\captionof{figure}{Kamada Kawai Layout}
	\end{center}
	
	\newpage
	\section{Random Layout}
	
	Este dise�o se utiliza para cualquier tipo de grafo conectado y no conectado, planos y no planos.Se caracteriza por darle un orden aleatorio a los nodos de un grafo y tiene como limitaci�n que para asegurarse de que los nodos no se superpongan con los margenes de la regi�n del dise�o,este algoritmo calcula las coordenadas al azar dentro de una regi�n con las dimensiones m�s peque�as qu la regi�n del dise�o \citep{RogueWave}. 
	\lstinputlisting[language=Python]{Tarea2_03.py}
	\begin{center}
		\includegraphics[scale=0.7]{Tarea2_03.eps}	
		\captionof{figure}{Random Layout}
	\end{center}
    \newpage
    \lstinputlisting[language=Python]{Tarea2_10.py}
    \begin{center}
    	\includegraphics[scale=0.7]{Tarea2_10.eps}	
    	\captionof{figure}{Random Layout}
    \end{center}
	
	\newpage
	\section{ForceAtlas2 Layout}
	
	En este dise�o los nodos se rechazan entre s�,mientras que los bordes lo atraen como resortes,la posici�n de cada nodo depende de los otros nodos \cite{journal.pone.0098679}.
	\lstinputlisting[language=Python]{Tarea2_4.py}
	\begin{center}
		\includegraphics[scale=0.7]{Tarea2_4.eps}	
		\captionof{figure}{ForceAtlas2 Layout}
	\end{center}

	
	\newpage
	\section{Shell Layout}
	
	Este dise�o posiciona los nodos en c�rculos conc�ntricos \citep{networkx}, basandose en la distancia entre el nodo central al resto de los nodos,por lo que los nodos se distribuyen de forma circular;este tiene como ventaja que su dise�o hace que sea un grafo visualmente de f�cil entender \citep{cherven2015mastering} 
		\lstinputlisting[language=Python]{Tarea2_05.py}
	\begin{center}
		\includegraphics[scale=0.7]{Tarea2_05.eps}	
		\captionof{figure}{Shell Layout}
	\end{center}
	
	\newpage
	\section{Kamada kawai Layout}
	
	Este algoritmo es dirigido por la fuerza entre dos nodos cualesquiera,donde los mismo se representan por anillos de acero y las aristas son los resortes entre ellos.Las fuerzas de atracci�n y de repulsi�n son an�logas a la fuerza de resorte,donde la suma de las fuerzas determinan en que direcci�n se debe mover un nodo \cite{fruchterman1991graph}; es utilizado en grafos no dirigidos muy grandes y garantiza que los nodos cercanos sean ubicados en la misma vencidad y los lejanos se ubican uno lejos de otros.
		\lstinputlisting[language=Python]{Tarea2_06.py}
		\begin{center}
			\includegraphics[scale=0.7]{Tarea2_06.eps}	
			\captionof{figure}{Kamada kawai Layout}
		\end{center}
	
	\newpage
	\section{Spring Layout}
	
	En este dise�o los nodos ser�an los cuerpos y los bordes los resortes que conectan a estos, los mismo son los que proporcionan fuerzas entre los nodos; estos se mueven de acuerdo con las fuerzas que se ejercen hasta alcanzar la m�nima energ�a local \cite{huck2014posicionamiento}.
		\lstinputlisting[language=Python]{Tarea2_07.py}
	\begin{center}
		\includegraphics[scale=0.7]{Tarea2_07.eps}	
		\captionof{figure}{Spring Layout}
	\end{center}
    \newpage
    	\lstinputlisting[language=Python]{Tarea2_11.py}
    \begin{center}
    	\includegraphics[scale=0.7]{Tarea2_11.eps}	
    	\captionof{figure}{Spring Layout}
    \end{center}

	 \newpage
	\section{Spectral Layout}
	
	Este dise�o utiliza los vectores de matrices relacionadas con gr�ficas como las de adyacencia, tiene como ventaja la capacidad de calcular dise�os �ptimos y presentar un tiempo de calculo muy r�pido \citep{brandes2007dynamic}.
	\lstinputlisting[language=Python]{Tarea2_08.py}
	    \begin{center}
		\includegraphics[scale=0.8]{Tarea2_08.eps}	
		\captionof{figure}{Spectral Layout}
	    \end{center}
    
	\newpage
	\section{Bipartite Layout}
	
	Estos grafos representan las relaciones entre dos conjuntos y permiten la exploraci�n paso a paso a gran escala.Los nodos de un mismo conjunto no se encuentran conectados entre s� \citep{ito2010drawing}.
		\lstinputlisting[language=Python]{Tarea2_09.py}
			\begin{center}
			\includegraphics[scale=0.7]{Tarea2_09.eps}	
			\captionof{figure}{Bipartite Layout}
		\end{center}
	    \newpage
	    \lstinputlisting[language=Python]{Tarea2_12.py}
	    \begin{center}
	    	\includegraphics[scale=0.7]{Tarea2_12.eps}	
	    	\captionof{figure}{Bipartite Layout}
	    \end{center}
			
	\newpage
	\bibliography{Bibliografia}
	\bibliographystyle{plainnat}
\end{document}