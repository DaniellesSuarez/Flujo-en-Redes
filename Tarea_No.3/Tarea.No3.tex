\documentclass[10pt,a4paper,openany]{article}
\usepackage[latin1]{inputenc}
\usepackage{amsmath}
\usepackage{amsfonts}
\usepackage{amssymb}
\usepackage{graphicx}
\usepackage{listings}
\usepackage{color}
\usepackage[left=2cm,right=2cm,top=3cm,bottom=2cm]{geometry}
\usepackage[numbers,sort&compress]{natbib}
\usepackage[spanish]{babel}
\usepackage{caption}
\usepackage{url}


\setlength{\parindent}{12pt}

\definecolor{mygreen}{rgb}{0,0.6,0}
\definecolor{mygray}{rgb}{0.5,0.5,0.5}
\definecolor{mymauve}{rgb}{0.58,0,0.82}

\lstset{ 
	backgroundcolor=\color{white},   % choose the background color; you must add \usepackage{color} or \usepackage{xcolor}; should come as last argument
	basicstyle=\footnotesize,        % the size of the fonts that are used for the code
	breakatwhitespace=false,         % sets if automatic breaks should only happen at whitespace
	breaklines=true,                 % sets automatic line breaking
	captionpos=b,                    % sets the caption-position to bottom
	commentstyle=\color{mygreen},    % comment style
	deletekeywords={...},            % if you want to delete keywords from the given language
	escapeinside={\%*}{*)},          % if you want to add LaTeX within your code
	extendedchars=true,              % lets you use non-ASCII characters; for 8-bits encodings only, does not work with UTF-8
	firstnumber=01,                	 % start line enumeration with line 1000
	frame=single,	                 % adds a frame around the code
	keepspaces=true,                 % keeps spaces in text, useful for keeping indentation of code (possibly needs columns=flexible)
	keywordstyle=\color{blue},       % keyword style
	language=Python,                 % the language of the code
	morekeywords={*,...},            % if you want to add more keywords to the set
	numbers=left,                    % where to put the line-numbers; possible values are (none, left, right)
	numbersep=5pt,                   % how far the line-numbers are from the code
	numberstyle=\tiny\color{mygray}, % the style that is used for the line-numbers
	rulecolor=\color{black},         % if not set, the frame-color may be changed on line-breaks within not-black text (e.g. comments (green here))
	showspaces=false,                % show spaces everywhere adding particular underscores; it overrides 'showstringspaces'
	showstringspaces=false,          % underline spaces within strings only
	showtabs=false,                  % show tabs within strings adding particular underscores
	stepnumber=1,                    % the step between two line-numbers. If it's 1, each line will be numbered
	stringstyle=\color{mymauve},     % string literal style
	tabsize=2,	                     % sets default tabsize to 2 spaces
	title=\lstname                   % show the filename of files included with \lstinputlisting; also try caption instead of title
}
\title{Tarea 3}
\author{5272}
\date{}


\begin{document}
	\maketitle
	
	
	\section{Algoritmos para grafos}
	\subsection{Todos los caminos m�s cortos}
	
	Se utiliza en grafos ponderados dirigidos conectados $G(V, E)$, donde para cada borde $<u,v> \in E$ se le asocia un peso $ w $ para el  problema de todos los pares de rutas m�s cortas. Tiene aplicaci�n en los problemas del servicio urbano, como la ubicaci�n de las instalaciones urbanas o la distribuci�n o entrega de bienes \citep{zwick2002all}. 
	\begin{center}
		\includegraphics[scale=0.7]{Tarea3_01.eps}	
		\captionof{figure}{}
	\end{center}
	
	\subsection{Centralidad intermedia}
	
	Este algoritmo calcula la ruta m�s corta ponderada entre cada par de nodos en un grafo conectado, en este se emplea la b�squeda por amplitud.Se utiliza para encontrar nodos que sirven de puente de una parte de un grafo a otra. Se aplica en un proceso de entrega de paquetes o red de telecomunicaciones \citep{white1994betweenness}.
	\begin{center}
		\includegraphics[scale=0.7]{Tarea3_02.eps}	
		\captionof{figure}{}
	\end{center}
    
    \subsection{B�squeda topol�gica}
    
    Este algoritmo crea un ordenamiento lineal de los v�rtices, de modo que si aparece el borde $(u,v)$ en el gr�fico, $v$ aparece antes que $u$ en el ordenamiento. El grafo debe de ser un gr�fico ac�clico dirigido \citep{pearce2007dynamic}.
    \begin{center}
    	\includegraphics[scale=0.7]{Tarea3_03.eps}	
    	\captionof{figure}{}
    \end{center}
    
    \subsection{Camarilla m�xima}
    
    Este algoritmo calcula la camarilla m�xima del grafo $G$, es decir, el subgrafo completo del tama�o m�ximo. El par�metro $ind$ es para la elecci�n del m�todo, si el par�metro es $0$ el m�todo es un algoritmo basado en la programaci�n cuadr�tica $0-1$. El tama�o de la salida es el n�mero de nodos de la camarilla encontrados por el algoritmo \cite{ostergaard2002fast}.
    \begin{center}
    	\includegraphics[scale=0.7]{Tarea3_04.eps}	
    	\captionof{figure}{}
    \end{center}
    
    \subsection{�rbol dfs }
    
    Este algoritmo de b�squeda de profundidad es una de las t�cnicas de desplazamiento gr�fica m�s utilizadas; implica atravesar el grafo y construir un �rbol de expansi�n (o bosque) arraigado \citep{baswana2017incremental}.
    \begin{center}
    	\includegraphics[scale=0.7]{Tarea3_05.eps}	
    	\captionof{figure}{}
    \end{center}
    
    \newpage
    \section{An�lisis de resultados}
    Luego de correr los 5 algoritmos para grafos diferentes con un tiempo computacional mayor a 5 segundos con un n�mero de replicas igual a 70 000 y realizar los histogramas correspondientes a cada uno, se demuestra que los datos no tienen una distribuci�n normal.
    
    Se realizan dos gr�ficos de dispersi�n Tiempo vs. Cantidad de nodos [Figura 6] y Tiempo vs. Cantidad de arista [Figura 7] donde se observa que el algoritmo m�s lento es el de Centralidad intermedia y el m�s r�pido es B�squeda topol�gica.
    
    \begin{center}
    	\includegraphics[scale=0.7]{Tarea3_06.eps}	
    	\captionof{figure}{Tiempo vs Cantidad de Nodos}
    \end{center}

    \begin{center}
    	\includegraphics[scale=0.7]{Tarea3_07.eps}	
    	\captionof{figure}{Tiempo vs Cantidad de Nodos}
    \end{center}

   
	\newpage
	\section{Fragmento de C�digo}
    \lstinputlisting[language=Python]{Tarea_3.py}
			
	\newpage
	\bibliography{Bibliografia}
	\bibliographystyle{plainnat}
\end{document}